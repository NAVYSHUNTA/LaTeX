% これは書籍、卒論などページ数が多いものを作成するときのテンプレートである。
% 1つのファイルに書くのではなく、ファイルを分割して\include{}で取り込む方式を採用した。
\documentclass[a4paper, 11pt]{jsbook}
\usepackage{amsmath, amssymb}  % 数式用
\usepackage{bm} % ベクトルなどの太字斜体用
\usepackage{listings} % ソースコード挿入用
\usepackage{graphicx} % 画像挿入用
\usepackage{subcaption} % 表や画像を横並びにする
\usepackage{wrapfig} % 表や画像の周りに文字を配置する
\usepackage[top=20truemm,bottom=20truemm,left=20truemm,right=20truemm]{geometry} % 余白の設定用
\setcounter{secnumdepth}{2}
\title{\LaTeX{}の使い方}
\author{Shunta Nakamura}
\date{\today}

\begin{document}
\maketitle
\tableofcontents
%\part{}
%\chapter{}
%\section{}
%\subsection{}
\part{\LaTeX{}の基礎} % 省略可
    \chapter{いろいろな文字・記号}
        \section{ギリシャ文字}
        \section{数式記号}
            \subsection{オイラーの公式を入力してみる}

%\include{chap1} % chap1.texファイルで作成した内容を取り込む
%\include{chap2} % chap2.texファイルで作成した内容を取り込む
%\include{chap3} % chap3.texファイルで作成した内容を取り込む
%\include{appendix1} % appendix1.texファイルで作成した内容を取り込む
%\include{appendix2} % appendix2.texファイルで作成した内容を取り込む
%\include{appendix3} % appendix3.texファイルで作成した内容を取り込む
\chapter*{謝辞} % \chapterに*を付けると章番号は省略される
\addcontentsline{toc}{chapter}{謝辞} % 目次に追加する

%\include{biblography} % 参考文献
\end{document}
